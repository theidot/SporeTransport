\documentclass[12pt]{amsart}
\usepackage{fullpage}

%------------------------------

\begin{document}

\author{Dave Schneider}
\address{Plant-Microbe Interaction Research Unit\\
  R.W.\ Holley Center \\
  USDA Agricultural Research Service \\
  538 Tower Road \\ Ithaca, NY 14583}

\title{An alternative approach to the ``conveyor belt'' model}

\email{Dave.Schneider@ars.usda.gov}

\begin{abstract}
  The Fisher-Kolmogorov equation
  \begin{displaymath}
    \left(\frac{\partial}{\partial t} + D\frac{\partial^2}{\partial
      x}\right)\theta = \theta(1-\theta)
  \end{displaymath}
  subject to some initial condition $\Phi(x, 0) = p(x)$ is often used
  as a starting point for the dispersal of a population with a
  logistic growth law.  It is well known that solutions to this
  equation evolving from initial conditions with compact support form
  uniformly translating fronts at long times and large distances.
  Extending this basic model has become a cottage industry by
  generalizing the differential operator (e.g., adding advection terms
  or introducing fraction derivatives) or alternative growth laws
  (e.g., introducing spatial dependence or stronger non-linearities).

  Motivated by the phenomenology of wind-driven dispersal of fungal
  plant pathogens, we propose an alternative two-layer model based on
  a separation of scales for transport.  The Fisher-Kolmogorov
  equation is used as a caricature of growth and short-distance
  dispersal due to winds at ground level.  A simple advection equation
  is used to model the long-distance transport in the upper
  atmosphere.  The two layers are coupled by means of rate constants
  characterizing entrainment of spores released at ground level and
  uplifted into the advection layer, and for deposition from the
  advection layer into the ground layer.

  As in the single-layer archetype, the two-layer model exhibits
  uniformly translating wave fronts.  However, the speed of the fronts
  in the two-layer model is different from that of fronts in a single layer
  model obtained by augmenting a Fisher-Kolomogorov equation with a
  advective term with the same ``wind speed''.  These differences can
  be traced back to a fundamental distinction in the two classes of models.  In
  the augmented single-layer model, the entire population is growing
  as it is being advected through space.  In contrast, in the
  two-layer model the growth of the population is limited to the
  ground layer.
\end{abstract}

\maketitle

%------------------------------

\section{Problem statement}

As provided in the draft manuscript (Equations P1 and P2).
\begin{subequations}
\begin{align}
  \frac{d\tilde{\rho}_i}{dt} & = -\nabla \cdot (\mathbf{v}\tilde{\rho}_{i}) +
  \alpha_{i} \tilde{\theta}_{i} -
  \beta_{i}\tilde{\rho}_{i} \label{eq:mass_balance_original}\\
  \frac{d\tilde{\theta}_i}{dt} & = \tilde{\delta}_{i}  \tilde{f}_i(\tilde{\theta}) -
  \alpha_{i} \tilde{\theta}_{i} +
  \beta_{i}\rho_i +
  \tilde{D}_{i}\nabla^2\theta_{i} \label{eq:growth_and_diffusion_original}
\end{align}
\end{subequations}
Note the appearance of total derivatives on the left hand sides and
partial derivatives on the right.  

The left hand side of Equation~\ref{eq:mass_balance_original} is the 
local conservation law for spore density in the upper layer while
the sum on the right hand side describes the net change of density to
exchange with the lower layer.  Similarily, the left hand side of
Equation~\ref{eq:growth_and_diffusion_original} describes the local diffusive
spread of spores within the lower layer while the terms on the right
describe density dependent growth and exchange with the upper layer.  

Since the manuscript only deals with the single species, a single
spatial dimension and a constant wind speed, it seems reasonable to
drop the subscripts, rescale variables and simplify the notation.
\begin{subequations}
\begin{align}
  \left(\frac{\partial}{\partial t} +
  \frac{\partial}{\partial x} +
  b\right)\rho &=
  a \theta \label{eq:balance_law}\\
  \left(\frac{\partial}{\partial t} - D\frac{\partial^2}{\partial x^2}+ a\right)\theta -
  f(\theta) & = b\rho \label{eq:Fisher_Kolmogorov_balance}
\end{align}
\end{subequations}

%------------------------------

\section{Qualitative structure}

The left hand side of Equation~\ref{eq:balance_law} defines a
linear partial differential operator describing advective flow and
deposition.  Equation~\ref{eq:Fisher_Kolmogorov_balance} is
semilinear --- the left hand side is a linear partial differential
operator and the non-linearity is due to the presence of the growth term 
on the right hand side~\cite{RozhdestvenskiiJanenko:1983}.

This system can be recast in the suggestive form
\begin{equation}
\frac{\partial u}{\partial t} + \frac{\partial F(u)}{\partial x} =
G\left[u\right]
\end{equation}
where $u = (\rho, \theta)^{\text{tr}}$ is a vector of densities, and
$F$ and $G$ are the associated fluxes and sources.
\begin{equation}
  \frac{\partial}{\partial t}
  \begin{pmatrix}
    \rho \\
    \theta
  \end{pmatrix}
  +
  \frac{\partial}{\partial x}
  \begin{pmatrix}
    1 & 0 \\
    0 & - D\frac{\partial\theta}{\partial x}
  \end{pmatrix}
  \begin{pmatrix}
    \rho \\
    \theta
  \end{pmatrix}
  =
  \begin{pmatrix}
    -b\rho + a\theta\\
    b\rho - a\theta + f(\theta)
  \end{pmatrix}
  \label{eq:densities_fluxes_sources}
\end{equation}
Note, $F(cu) = cF(u)$ consistent with the semilinear character of the
equations.

Equations of this type have been extensively studied based on the
generalized notions of
entropy~\cite{Lax:1957,FriedrichsLax:1971,BoillatRuggeri:1997,Yong:2008}
or the theory of semigroups for linear differential
equations~\cite{Segal:1963,OharuTakahashi:1989,Miyadera:1992,CazenaveHaraux:1998,Bellini-MoranteMcBride:1998,KawashimaYong:2004,KobayashiMatsumotoTanaka:2007}.

Semigroups, sometimes called propagators, are often used to describe
the time evolution of systems governed by linear differential
equations~\cite{Pazy:1983,EngelNagel:2000}.  Much of this theory can
been generalized to semilinear
equations~\cite{Miyadera:1992,CazenaveHaraux:1998,Bellini-MoranteMcBride:1998}.
The most important conclusion is that the exact solution can be
written as
\begin{equation}
  u(x,t) = e^{Lt} u(x, 0) + \int_{0}^{t} e^{L(t-s)}N\left[u(x, s)\right] \; ds
  \label{eq:nonlinear_semigroup}
\end{equation}
where $L$ is the generator defined by the linear part of the problem
and $N$ is the nonlinear part.  The existence of the semigroup and
it's precise properties depended on the domain of the linear
differential operator obtained by splitting the problem into linear
and non-linear parts, and the smoothness of the non-linear part.  This
splitting is not unique and can be chosen to simplify calculations.
Equation~\ref{eq:nonlinear_semigroup} shows that the non-linear parts
of semilinear PDEs play roles somewhat analogous to inhomogeneous
terms in the theory of linear PDEs.  In particular, the cumulative
effects of the linear and non-linear terms are additive.

\subsection{Initial conditions}

The paper focus on the solution of Cauchy (initial-boundary value)
problems for several specific initial conditions, but does not
describe the space of allowed initial values.  The specific initial
conditions in the paper all assume $\rho(x, 0) = 0$.  Two initial
conditions for $\theta$ are discussed
\begin{equation}
  \theta(x, 0) =
  \begin{cases}
    \delta(x) & \text{Point source} \\
    e^{-c\left|x\right|} & \text{Exponential tails}
  \end{cases}
\end{equation}
The point source case suggests the need to interpret derivatives in
the sense of distributions.

\subsection{Space of solutions}

As noted above, the semigroup approach involves a careful
specification of the domain of the generator and related issues of
continuity and convergence.  To begin, one must define the space of
allowed solutions, $\mathsf{X}$.  Clearly, $\mathsf{X}$ must be some
sort of topological vector space over $\mathbb{R}$.

Note, valid solutions to Equation~\ref{eq:densities_fluxes_sources}
involve non-negative densities $\rho(x, t) \ge 0$, $\theta(x, t) > 0$,
so it makes sense to equip $\mathsf{X}$ with the partial order relation,
\begin{equation}
  \phi_1(x, t) \le \phi_2(x, t), \psi_1(x, t) \le \psi_2(x, t) \iff
  \begin{pmatrix}
    \phi_1 \\
    \psi_1
  \end{pmatrix} \le
  \begin{pmatrix}
    \phi_2 \\
    \psi_2
  \end{pmatrix}
\end{equation}
This partial order on $\mathsf{X}$ is clearly translational invariant, positive
homogeneous, and inherits meet and join operations defined by the
pointwise $\max$ and $\min$ operations on the underlying densities.
Thus, $\mathsf{X}$ can be viewed as a vector lattice, and valid solutions
are elements of the positive cone.

The theory of non-linear semigroups is based on the additional
assumption that $\mathsf{X}$ is a Banach space -- a vector space that is
complete with respect to the topology induced by a norm.  Adopting
this approach involves re-interpreting $u$ as a $\mathsf{X}$-valued function of
time in some interval $t\in T=\left[t_0, t_1\right]$
\begin{displaymath}
 u : T \rightarrow X, t \rightarrow u(t) 
\end{displaymath}
so that Equation~\ref{eq:densities_fluxes_sources} becomes an ordinary
differential equation in $\mathsf{X}$.  The maximal time interval over
which $u$ can be defined is related to the fact that
Equation~\ref{eq:densities_fluxes_sources} is an initial value
problem.  There is no loss of generality in assuming $t_0=0$ since
Equation~\ref{eq:densities_fluxes_sources} is autonomous.  The upper
end of the interval is limited by the possibility that $u$ might
``escape'' $\mathsf{X}$ at $t=t_1$
\begin{equation}
  \lim_{t\uparrow t_1} \left\|u(t)\right\|_{\mathsf{X}} = \infty
  \label{eq:u_escapes_X}
\end{equation}

The problem at hand presents an embarrassment of riches -- the
existence of many norms motivated by different aspects of the
underlying biology processes and physical considerations.  The $L^{1}$
norm arises from the real world restriction that the total population
cannot become infinite
\begin{equation}
  \left\|u(t)\right\|_{L^{1}} = \int_{-\infty}^{\infty} \rho(x, t) +
  \theta(x, t) \; dx < \infty
\end{equation}
Similiar common-sense considerations suggest that the total density
must be finite almost everywhere,
\begin{equation}
  \left\|u(t)\right\|_{L^{\infty}} = \sup_{-\infty < x < \infty}
  \rho(x, t) + \theta(x, t) < \infty
\end{equation}
The next step is to consider the fluxes.  Introducing a Sobolev norm 
involving first and second order spatial derivatives
would guarantee that the total flux is finite
\begin{equation}
  \left\|u(t)\right\|_{W^{2,1}} = \left\|u\right\|_{L^{1}} + 
  \left\|\frac{\partial u}{\partial x}\right\|_{L^{1}} +
  \left\|\frac{\partial^2 u}{\partial x^2}\right\|_{L^{1}} < \infty  
\end{equation}
or is finite almost everywhere
\begin{equation}
  \left\|u(t)\right\|_{W^{2,\infty}} =
  \max\left\{\left\|u\right\|_{L^{\infty}}, 
  \left\|\frac{\partial u}{\partial x}\right\|_{L^{\infty}},
  \left\|\frac{\partial^2 u}{\partial
    x^2}\right\|_{L^{\infty}}\right\} < \infty
\end{equation}
The $W^{2, 1}$ and $W^{2, \infty}$ norms do not exclude the presence
of a finite number of finite discontinuities in the solution or its
derivatives.

The complete specification of the problem must include the
definition of boundary conditions at $x=\pm\infty$, and issue not
explicitly addressed anywhere in the manuscript.  However, the
following assumptions are used implicitly
\begin{equation}
  0 = 
  \lim_{x\rightarrow \pm\infty} \rho(x, t) =
  \lim_{x\rightarrow \pm\infty} \frac{\partial\rho}{\partial x} =  
  \lim_{x\rightarrow \pm\infty} \theta(x, t)  =  
  \lim_{x\rightarrow \pm\infty} \frac{\partial \theta}{\partial x} = 
  \lim_{x\rightarrow \pm\infty} \frac{\partial^2 \theta}{\partial x^2}
  \label{eq:boundary_conditions}
\end{equation}


Finally, one must address the source term, $G\left[u\right]$, on the
right hand side of Equation~\ref{eq:densities_fluxes_sources}.  The
simplest case is to choose a norm such that $G: X \rightarrow X$ is
bounded.  

The selection of a norm, or set of norms, to use in the analysis of
various aspects of the problem is simplified by exploiting the known
relationships among the various norms and the effect of imposing the
boundary conditions (cf.\ Equation~\ref{eq:boundary_conditions}).  For
example, it is be shown~\cite[Theorem 8.1.1]{Strichartz:2003} that
if $f\in W^{2,1}(\mathbb{R})$ implies
\begin{equation}
  \left\|f\right\|_{L^{\infty}} \le c \left\|f\right\|_{W^{2,1}}
\end{equation}
for some constant $c$.  In fact, by identifying continuous functions
with their equivalence classes,
\begin{equation}
  W^{2,1}(\mathbb{R}) \subseteq C^{1}(\mathbb{R})
\end{equation}


The latter choice has many practical advantages since
$L^{1}_{\text{loc}}(\mathbb{R}) \subset \mathcal{D}'(\mathbb{R})$
where $\mathcal{D}'(\mathbb{R})$ is the space of distributions on
$\mathbb{R}$.  Thus, if $h\in L^{1}_{\text{loc}}(\mathbb{R})$ then if
\begin{equation}
  \int_{-\infty}^{\infty} h\phi \; dx = 0
\end{equation}
for all test functions $\phi\in \mathcal{D}(\mathbb{R}) =
C^{\infty}_{0}(\mathbb{R})$ implies $h=0$ by Du Bois-Reymond's
lemma~\cite{Grubb:2009}.

Also, The boundary conditions are not explicit stated anywhere in the
manuscript.  The boundary conditions
\begin{equation}
  0 = 
  \lim_{x\rightarrow \pm\infty} \rho(x, t) =
  \lim_{x\rightarrow \pm\infty} \frac{\partial\rho}{\partial x} =  
  \lim_{x\rightarrow \pm\infty} \theta(x, t)  =  
  \lim_{x\rightarrow \pm\infty} \frac{\partial \theta}{\partial x}
\end{equation}
will be assumed here.  




By using standard methods from the theory of distributions one can
proceed indirectly by considering the action of the differential
operators on elements of $\mathcal{D}(\mathbb{R})$ and inferring the
action on elements of $L^{1}_{\text{loc}}(\mathbb{R})$ using
``integration by parts''.  Note also that non-trivial constant
functions do not satisfy the boundary conditions stated above.



\subsection{Solution to homogeneous balance law}

It is straightforward to verify that the
solution to the generalized balance law 
\begin{equation}
  \left(\frac{\partial}{\partial y} + 
  \frac{\partial}{\partial z} + c\right)g(y, z) = 0
  \label{eq:homogeneous_conservation_law}
\end{equation}
for initial condition $g(y, 0) = h(y)$ is~\cite[Equation
  4.2.1.1]{PolyaninSaitsevMoussiaux:2002} are waves moving at a
constant speed that preserve the shape but not the overall amplitude
of the initial density profile
\begin{equation}
  g(y, z) = e^{-cy}h(y-z)
  \label{eq:homogeneous_conservation_solution}
\end{equation}
 
%------------------------------

\subsection{Growth law}

The growth law is a continuously Frech\'et differentiable function $f \in
C^{1}\left(\mathbb{R}_{+}, \mathbb{R}\right)$
\begin{itemize}
  %
\item Linear growth at low density
  \begin{align}
    f(\theta=0) &= 0 \\
    f'(\theta = 0) &= 1
  \end{align}
  %
\item Convexity
  \begin{equation}
    rf(\theta_1) + (1-r)f(\theta_2) \ f\left(r\theta_1 + (1-r)\theta_2\right)
  \end{equation}
  for $0 < r < 1$.
  %
\item Vanishing at $\theta_{\text{max}}$
  \begin{equation}
    f(\theta_{\text{max}}) = 0
  \end{equation}
  %
\end{itemize}

It follows that $f'(\theta)$ is a decreasing function and $f(\theta) <
0$ for $\theta>\theta_{\text{max}}$.  

%------------------------------

\section{Conservation law}

Adding Equations~\ref{eq:balance_law}
and~\ref{eq:Fisher_Kolmogorov_balance} yield a relationship for
the rate of change of total amount of material in the system.
\begin{equation}
  \frac{\partial\rho}{dt} + \frac{\partial\theta}{dt} =
  \frac{\partial \rho}{\partial x} +
  \mathcal{D}\frac{\partial^2\theta}{\partial x^2} +
  f(\theta)
\end{equation}
Note, this equation depends on $\mathcal{D}$,
but not $a$ or $b$.  Integrating both sides over all $x$
\begin{align}
  \int_{-\infty}^{\infty} \frac{\partial}{\partial t} \left[\rho(x,
    t)+\theta(x,t)\right]\; dx &=
  \int_{-\infty}^{\infty} f(\theta) \; dx +
  \int_{-\infty}^{\infty} \frac{\partial}{\partial x} \left[-\rho +
    \mathcal{D}\frac{\partial \theta}{\partial x}\right]\; dx \\
  \frac{\partial}{\partial t} \int_{-\infty}^{\infty}
  \left[\rho(x, t)+\theta(x,t)\right]\; dx &=
  \int_{-\infty}^{\infty} f(\theta) \; dx + \left.\left[-\rho +
    \mathcal{D}\frac{\partial \theta}{\partial
      x}\right]\right|_{x=-\infty}^{x=+\infty}
 \end{align}
The contribution of the transport term vanishes due to the boundary
conditions leaving a non-linear conservation law
\begin{equation}
  \frac{\partial}{\partial t} \int_{-\infty}^{\infty}
  \left[\rho(x, t)+\theta(x,t)\right]\; dx = \int_{-\infty}^{\infty}
  f(\theta) \; dx
  \label{eq:general_conservation_law}
\end{equation}
in which none of the parameters appear explicitly.  This is a
common-sense statement -- if a spore has been produced it must be
found somewhere in the air or on the ground, and if it is on the
ground then it contributes to further growth.

If $\theta(x, t) < \theta_{\max}$ then the right-hand side of
Equation~\ref{eq:general_conservation_law} is positive and the total
mass increases with time.

%------------------------------

\section{Properties of the solution for $\mathcal{D}=0$}

Setting $\mathcal{D}=0$ in
Equation~\ref{eq:Fisher_Kolmogorov_balance} leaves
the simplified dynamical equation for the growth layer
\begin{equation}
  \left(\frac{\partial}{\partial t}+a\right)\theta
  - f(\theta) = b\rho\label{eq:growth_alone}
\end{equation}
The dependence on $\rho$ can be eliminated using Equation~\ref{eq:balance_law}
\begin{align}
  \left(\frac{\partial}{\partial t} + \frac{\partial}{\partial x} + b\right)\left(
  \frac{\partial}{\partial t}+a\right)\theta -
  \left(\frac{\partial}{\partial t} + \frac{\partial}{\partial x} + b\right)f(\theta) &
  = b\left(\frac{\partial}{\partial t} + \frac{\partial}{\partial x} + b\right)\rho \\ &
  = ab\theta
  \label{eq:fixed_point_formulation}
\end{align}
or, by rearrangement,
\begin{align}
  \left(\frac{\partial}{\partial t} + \frac{\partial}{\partial x} + b\right)\left(
  \frac{\partial}{\partial t}+a\right)\theta - ab\theta
  = \left(\frac{\partial}{\partial t} + \frac{\partial}{\partial x} + b\right)f(\theta) &
  \label{eq:linear_nonlinear_balance}
\end{align}
Note, the left hand side is a linear, second order partial
differential equation in $\theta$ with constant coefficients
describing the net effect of the transport processes while the right
hand side is a linear, first order differential equation in
$f(\theta)$ with constant coefficients describing a damped wave in the
growth rate.


%------------------------------

\section{Characteristic values for linearized system with logistic growth and $D=0$}

For logistic growth, $f(\theta) = \theta(1-\theta)$ so
\begin{equation}
  \left(\frac{\partial}{\partial t} + \frac{\partial}{\partial x} + b\right)f(\theta) = 
\left(1-2\theta\right)\left(\frac{\partial}{\partial t} + \frac{\partial}{\partial x} + b\right)\theta 
\end{equation}
Linearizing this expression seems plausible in regions where $\theta
\ll 1/2$ so

\begin{align}
  \left(\frac{\partial}{\partial t} + \frac{\partial}{\partial x} + b\right)\left(
  \frac{\partial}{\partial t}+a\right)\theta - ab\theta
  = \left(\frac{\partial}{\partial t} + \frac{\partial}{\partial x} + b\right)\theta
  \label{eq:linearized_logistic_theta}
\end{align}

Inserting the expression for $\theta$ as an inverse Fourier transform
\begin{equation}
  \theta(x, t) = \int_{-\infty}^{\infty}\int_{-\infty}^{\infty} A(k, s) e^{i(st-kx)} \; ds dk
\end{equation}
into Equation~\ref{eq:linearized_logistic_theta} yields the characteristic equation
\begin{equation}
(is-ik+b)(is+a) - ab = (is-ik+b) 
\end{equation}
or, equivalently, 
\begin{equation}
  \left[b+i(s-k)\right](is+a-1) - ab = -s^2 + \left[k+i(a+b-1)\right]s -
  \left[b + ik(a-1)\right]
\end{equation}
which has two solutions 
\begin{align}
  s_{\pm}(k)
  & = \frac{k + i(a+b-1)}{2} \pm \frac{\sqrt{\left[k + i(a+b-1)\right]^2-4\left[b+ik(a-1)\right]}}{2} \\
  & = \frac{k + i(a+b-1)}{2} \pm \frac{\sqrt{\left[k + i(1-a+b)\right]^2-4ab}}{2}
  \label{eq:dispersion_relation_combined}
\end{align}
This form of the dispersion relation is somewhat unsatisfactory
because neither solution is single valued as a function of $k$ with
the usual choice of branch cut along the negative real axis.  This 
situation can be resolved by redefining $s_{\pm}$ as
\begin{equation}
  s_{\pm}(k) = \frac{k + i(a+b-1)}{2} \pm \frac{1}{2}\sqrt{k - k_1}\sqrt{k - k_2} \\
  \label{eq:dispersion_relation_separated}
\end{equation}
where
\begin{align}
  k_1 & = -i(1-a+b)-2\sqrt{ab} \\
  k_2 & = -i(1-a+b)+2\sqrt{ab}   
\end{align}
and the usual branch cut definition is used for all square roots.
Note, this is equivalent to defining a branch cut between $k_1$ and
$k_2$ for the square root term in appearing in
Equation~\ref{eq:dispersion_relation_combined}.

Since Equation~\ref{eq:dispersion_relation_separated} coincides with
the dispersion relation given in the draft paper, the corresponding
results carry over in their entirety.

%------------------------------

\section{Non-linear dynamics}

Consider the full non-linear equation for 
\begin{equation}
G\left[\theta\right] = \left(\frac{\partial}{\partial t} + \frac{\partial}{\partial x} +
  b\right)\left( \frac{\partial\theta}{\partial t}
  -D\frac{\partial^2\theta}{\partial x^2}- f(\theta) +
  a\theta\right) = ab\theta
  \label{eq:full_non_linear_theta}
\end{equation}
One interpretation of this equation is that the solution is a fixed
point of the iteration
\begin{equation}
  \theta_{k+1} = \frac{1}{ab}G\left[\theta_{k}\right]
  \label{eq:fixed_point_iteration}
\end{equation}
However, it is not obvious that this sequence will converge.

To begin, consider the convex linear combination of two non-negative densities,
$\eta_1$ and $\eta_2$,
\begin{equation}
  \phi(x, t) = r\eta_1(x, t)+(1-r)\eta_2(x, t) 
\end{equation}
From Equations~\ref{eq:fixed_point_formulation}
and~\ref{eq:fixed_point_iteration} it follows that
\begin{equation}
G[\phi] = rG[\eta_1] + (1-r)G[\eta_2] - \left(\frac{\partial}{\partial t} + \frac{\partial}{\partial x} +
b\right)\left[f\left(\phi\right) - rf(\eta_1)-(1-r)f(\eta_2)\right]
\end{equation}
Recall that $f$ was assumed to be convex
\begin{equation}
rf(\eta_1)+(1-r)f(\eta_2) \le
f\left(r\eta_1+(1-r)\eta_2\right) = f(\phi)
\end{equation}
which implies
\begin{equation}
  \psi(x, t) = f\left(\phi\right) - rf(\eta_1)-(1-r)f(\eta_2) \ge 0
\end{equation}
and
\begin{equation}
  G\left[r\eta_1+(1-r)\eta_2\right] - rG\left[\eta_1\right] - (1-r)
  G\left[\eta_2\right] = - \left(\frac{\partial}{\partial t} +\frac{\partial}{\partial x} +
b\right)\left[f\left(\phi\right) - rf(\eta_1)-(1-r)f(\eta_2)\right]
\end{equation}
%------------------------------

\bibliographystyle{unsrt}
\bibliography{biotic_transport}

\end{document}
