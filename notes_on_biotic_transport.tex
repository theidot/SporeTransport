\documentclass[12pt]{amsart}
\usepackage{fullpage}

\begin{document}

\section{Problem statement}

As provided in the draft manuscript (Equations P1 and P2).
\begin{align}
  \frac{d\tilde{\rho}_i}{dt} & = -\nabla \cdot (\mathbf{v}\tilde{\rho}_{i}) +
  \alpha_{i} \tilde{\theta}_{i} -
  \beta_{i}\tilde{\rho}_{i} \label{eq:mass_balance_original}\\
  \frac{d\tilde{\theta}_i}{dt} & = \tilde{\delta}_{i}  \tilde{f}_i(\tilde{\theta}) -
  \alpha_{i} \tilde{\theta}_{i} +
  \beta_{i}\rho_i +
  D_{i}\nabla^2\theta_{i} \label{eq:growth_and_diffusion_original}
\end{align}
Note the appearance of total derivatives on the left hand sides and
partial derivatives on the right.  

The left hand side of Equation~\ref{eq:mass_balance_original} is the 
local conservation law for spore density in the upper layer while
the sum on the right hand side describes the net change of density to
exchange with the lower layer.  Similarily, the left hand side of
Equation~\ref{eq:growth_and_diffusion_original} describes the local diffusive
spread of spores within the lower layer while the terms on the right
describe density dependent growth and exchange with the upper layer.  

Since the manuscript only deals with the single species, a single
spatial dimension and a constant wind speed, it seems reasonable to
drop the subscripts, rescale variables and simplify the notation.  
\begin{align}
  \left(\frac{\partial}{\partial t} +
  \frac{\partial}{\partial x} +
  b\right)\rho &=
  a \theta \label{eq:balance_law}\\
  \left(\frac{\partial}{\partial t} - \mathcal{D}\frac{\partial^2}{\partial x^2}+ a\right)\theta -
  f(\theta) & = b\rho \label{eq:inhomogeneous_Fisher_Kolmogorov}
\end{align}
Note, the left hand side of Equation~\ref{eq:balance_law} defines a
linear partial differential operator describing advective flow and
deposition.  Equation~\ref{eq:inhomogeneous_Fisher_Kolmogorov} is
semilinear --- the non-linearity is confined to the zeroth order
term~\cite{RozhdestvenskiiJanenko:1983}.

\subsection{Boundary conditions and space of solutions}

The boundary conditions are not explicit stated anywhere in the
manuscript.  The boundary conditions
\begin{equation}
  \lim_{x\rightarrow \pm\infty} \theta(x, t) = 0 =
  \lim_{x\rightarrow \pm\infty} \rho(x, t) 
\end{equation}
will be used here.

Similarly, the class of allowed solutions is not defined.  In this
note, it is assumed that the $\rho(\cdot, t)$ and $\theta(\cdot,
t)$ are elements in $C^{\infty}(\mathbb{R})$ for every $t$.

\subsection{Solution to homogeneous balance law}

It is straightforward to verify that the
solution to the generalized balance law 
\begin{equation}
  \left(\frac{\partial}{\partial y} + 
  \frac{\partial}{\partial z} + c\right)g(y, z) = 0
  \label{eq:homogeneous_conservation_law}
\end{equation}
for initial condition $g(y, 0) = h(y)$ is~\cite[Equation 4.2.1.1]{PolyaninSaitsevMoussiaux:2002}
\begin{equation}
  g(y, z) = e^{-cz}h(y-z)
  \label{eq:homogeneous_conservation_solution}
\end{equation}
The natural time scale for relaxation is $1/c$ and the length scale is
determined by the initial conditions.

In this case, let
\begin{align}
  \eta &= x+t \\
  \zeta &= sx+(1-s)t
\end{align}
where $0\le s\le 1$ is a parameter to be chosen later.
This transformation is non-singular
\begin{equation}
  J(y, z; \eta, \zeta) =
  \begin{pmatrix}
    \frac{\partial\eta}{\partial t} & \frac{\partial \eta}{\partial x} \\
    \frac{\partial\zeta}{\partial t} & \frac{\partial\zeta}{\partial x} 
  \end{pmatrix}
  =
  \begin{pmatrix}
    -1 & 1 \\
    1-s & s 
  \end{pmatrix}
\end{equation}
thus $\det(J) = -s - (1-s) = -1$.

\begin{align}
  \frac{\partial}{\partial x} & = \frac{\partial\eta}{\partial x}\frac{\partial}{\partial\eta} +
  \frac{\partial\zeta}{\partial x}\frac{\partial}{\partial\zeta} \\
  &= \frac{\partial}{\partial\eta} + s \frac{\partial}{\partial\zeta} \\
  \frac{\partial}{\partial t} & = \frac{\partial\eta}{\partial t}\frac{\partial}{\partial\eta} +
  \frac{\partial\zeta}{\partial t}\frac{\partial}{\partial\zeta}\\
  &= -\frac{\partial}{\partial\eta} + (1-s) \frac{\partial}{\partial\zeta} 
\end{align}

Let $\tilde{g}(\eta, \zeta) = g(x, t)$
\begin{equation}
  \left(\frac{\partial}{\partial\zeta} + c\right)\tilde{g}(\eta, \zeta) = 0
\end{equation}
so
\begin{equation}
\tilde{g}(\eta, \zeta) = e^{-c\zeta}\phi(\eta)  
\end{equation}
where $\phi(\eta)$ is an arbitrary smooth function.  Note, this equation preserves positivity,
\begin{equation}
  \phi(\zeta)\ge 0 \implies \tilde{g}(\eta, \zeta)=g(x, t)\ge 0
\end{equation}

\subsection{System in transformed coordinates}

Let $\tilde{\rho}(\xi, \eta) = \rho(x, t)$ and $\tilde{\theta} = \theta(x, t)$.  Since
\begin{align}
  \frac{\partial^2}{\partial x^2} & = \left(\frac{\partial}{\partial\eta} + s \frac{\partial}{\partial\zeta}\right)\left(\frac{\partial}{\partial\eta} + s \frac{\partial}{\partial\zeta}\right) \\
  &= \frac{\partial^2}{\partial\eta^2} + 2s \frac{\partial^2}{\partial\eta\partial\zeta} + s^2\frac{\partial^2}{\partial\zeta^2}
\end{align}
the transformation of Equation~\ref{eq:inhomogeneous_Fisher_Kolmogorov} is simplified by choosing $s=0$.
Performing these substitutions yields the transformed equations
\begin{align}
  \left(\frac{\partial}{\partial\zeta}+b\right)\tilde{\rho} &= a\tilde{\theta} \label{eq:transformed_balance_law}\\
  \left(\frac{\partial}{\partial\zeta}-\frac{\partial}{\partial\eta}-
  \mathcal{D}\frac{\partial^2}{\partial\eta^2}+a\right)\tilde{\theta} + f(\tilde{\theta}) &= b\tilde{\rho}
  \label{eq:transformed_inhomogeneous_Fisher_Kolmogorov}
\end{align}

\subsection{Growth law}

The growth law is a continuously differentially function $f \in C^{1}\left(\mathbb{R}_{+}, \mathbb{R}\right)$
\begin{itemize}
  %
\item Linear growth at low density
  \begin{align}
    f(\theta=0) &= 0 \\
    f'(\theta = 0) &= 1
  \end{align}
  %
\item Convexity
  \begin{equation}
    \frac{f(\theta_1) + f(\theta_2)}{2} < f\left(\frac{\theta_1 + \theta_2}{2}\right)
  \end{equation}
  %
\item Vanishing at $\theta_{\text{max}}$
  \begin{equation}
    f(\theta_{\text{max}}) = 0
  \end{equation}
  %
\end{itemize}

It follows that $f'(\theta)$ is a decreasing function and $f(\theta) <
0$ for $\theta>\theta_{\text{max}}$.  

%------------------------------

\section{Conservation law}

Adding Equations~\ref{eq:balance_law}
and~\ref{eq:inhomogeneous_Fisher_Kolmogorov} yield a relationship for
the rate of change of total amount of material in the system.
\begin{equation}
  \frac{\partial\rho}{dt} + \frac{\partial\theta}{dt} =
  \frac{\partial \rho}{\partial x} +
  \mathcal{D}\frac{\partial^2\theta}{\partial x^2} +
  f(\theta)
\end{equation}
Note, this equation depends on $\mathcal{D}$,
but not $a$ or $b$.  Integrating both sides over all $x$
\begin{align}
  \int_{-\infty}^{\infty} \frac{\partial}{\partial t} \left[\rho(x,
    t)+\theta(x,t)\right]\; dx &=
  \int_{-\infty}^{\infty} f(\theta) \; dx +
  \int_{-\infty}^{\infty} \frac{\partial}{\partial x} \left[-\rho +
    \mathcal{D}\frac{\partial \theta}{\partial x}\right]\; dx \\
  \frac{\partial}{\partial t} \int_{-\infty}^{\infty}
  \left[\rho(x, t)+\theta(x,t)\right]\; dx &=
  \int_{-\infty}^{\infty} f(\theta) \; dx + \left.\left[-\rho +
    \mathcal{D}\frac{\partial \theta}{\partial
      x}\right]\right|_{x=-\infty}^{x=+\infty}
 \end{align}
The contribution of the transport term vanishes due to the boundary
conditions leaving a non-linear conservation law
\begin{equation}
  \frac{\partial}{\partial t} \int_{-\infty}^{\infty}
  \left[\rho(x, t)+\theta(x,t)\right]\; dx = \int_{-\infty}^{\infty}
  f(\theta) \; dx
  \label{eq:general_conservation_law}
\end{equation}
in which none of the parameters appear explicitly.  This is a
common-sense statement -- if a spore has been produced it must be
found somewhere in the air or on the ground, and if it is on the
ground then it contributes to further growth.

If $\theta(x, t) < \theta_{\max}$ then the right-hand side of
Equation~\ref{eq:general_conservation_law} is positive and the total
mass increases with time.

%------------------------------

\section{Properties of the solution for $\mathcal{D}=0$}

Setting $\mathcal{D}=0$ in
Equation~\ref{eq:transformed_inhomogeneous_Fisher_Kolmogorov} leaves
the simplified dynamical equation for the growth layer
\begin{equation}
  \left(\frac{\partial}{\partial\zeta}-\frac{\partial}{\partial\eta}+a\right)\tilde{\theta}
  + f(\tilde{\theta}) = b\tilde{\rho}\label{eq:growth_alone}
\end{equation}
The dependence on $\rho$ can be eliminated using Equation~\ref{eq:transformed_balance_law}
\begin{align}
  \left(\frac{\partial}{\partial\zeta} + b\right)\left(
  \frac{\partial}{\partial\zeta}
  -\frac{\partial}{\partial\eta}+a\right)\tilde{\theta} -
  \left(\frac{\partial}{\partial\zeta} + b\right)f(\tilde{\theta}) &
  = b\left(\frac{\partial}{\partial\zeta} + b\right)\tilde{\rho} \\ &
  = ab\tilde{\theta}
  \label{eq:fixed_point_formulation}
\end{align}
or, by rearrangement,
\begin{align}
  \left(\frac{\partial}{\partial\zeta} + b\right)
  \left(\frac{\partial}{\partial\zeta}-\frac{\partial}{\partial\eta}+a\right)\tilde{\theta}
  -ab\tilde{\theta} & = \left(\frac{\partial}{\partial\zeta} + b\right)f(\tilde{\theta}) \\
    \left(
    \frac{\partial^2}{\partial\zeta^2} -
    \frac{\partial^2}{\partial\eta\partial\zeta} +
    (a+b)\frac{\partial}{\partial\zeta} -
    b\frac{\partial}{\partial\eta} 
  \right)\tilde{\theta} &=
\end{align}
Note, the left hand side is a linear, second order partial
differential equation in $\tilde{\theta}$ with constant coefficients
describing the net effect of the transport processes while the right
hand side is a linear, first order differential equation in
$f(\tilde{\theta})$ with constant coefficients describing a damped wave in the
growth rate.



%------------------------------

\section{Characteristic values for linearized system with logistic growth and $D=0$}

For logistic growth, $f(\theta) = \theta(1-\theta)$ so
\begin{equation}
  \left(\frac{\partial}{\partial\zeta} + b\right)f(\tilde{\theta}) = 
  \left(\frac{\partial}{\partial\zeta} +b \right)\tilde{\theta} -
  \left(\frac{\partial}{\partial\zeta} + b\right)\tilde{\theta}^2
\end{equation}
implies
\begin{equation}
    \left(
    \frac{\partial^2}{\partial\zeta^2} -
    \frac{\partial^2}{\partial\eta\partial\zeta} +
    (a+b-1)\frac{\partial}{\partial\zeta} -
    b\frac{\partial}{\partial\eta}-b
  \right)\tilde{\theta} =  - \left(\frac{\partial}{\partial\zeta} + b\right)\tilde{\theta}^2
\end{equation}

The right hand side will be ignored near $\tilde{\theta}=0$ to yield the linearized equation
\begin{equation}
    \left(
    \frac{\partial^2}{\partial\zeta^2} -
    \frac{\partial^2}{\partial\eta\partial\zeta} +
    (a+b-1)\frac{\partial}{\partial\zeta} -
    b\frac{\partial}{\partial\eta}-b
    \right)\tilde{\theta} = 0
  \label{eq:linearized_logistic_theta}    
\end{equation}

Inserting the expression $\tilde{\theta}$ as an inverse Fourier transform
\begin{equation}
  \tilde{\theta}(\eta, \zeta) = \int_{-\infty}^{\infty}\int_{-\infty}^{\infty} A(k, s) e^{i(s\zeta-k\eta)} \; ds dk
\end{equation}
into Equation~\ref{eq:linearized_logistic_theta} yields the characteristic equation
\begin{align}
  (-is)^2-(is)(-ik)+(a+b-1)(is)-b(-ik)-b &= 0 \\
  -s^2+\left[k+i(a+b-1)\right]s -b(1+ik) 
\end{align}
which has two solutions 
\begin{align}
  s_{\pm}(k)
  & = \frac{k + i(a+b-1)}{2} \pm \frac{\sqrt{\left[k + i(a+b-1)\right]^2-4\left[b+ik(a-1)\right]}}{2} \\
  & = \frac{k + i(a+b-1)}{2} \pm \frac{\sqrt{\left[k + i(1-a+b)\right]^2-4ab}}{2}
  \label{eq:dispersion_relation_combined}
\end{align}
This form of the dispersion relation is somewhat unsatisfactory
because neither solution is single valued as a function of $k$ with
the usual choice of branch cut along the negative real axis.  This 
situation can be resolved by redefining $s_{\pm}$ as
\begin{equation}
  s_{\pm}(k) = \frac{k + i(a+b-1)}{2} \pm \frac{1}{2}\sqrt{k - k_1}\sqrt{k - k_2} \\
  \label{eq:dispersion_relation_separated}
\end{equation}
where
\begin{align}
  k_1 & = -i(1-a+b)-2\sqrt{ab} \\
  k_2 & = -i(1-a+b)+2\sqrt{ab}   
\end{align}
and the usual branch cut definition is used for all square roots.
Note, this is equivalent to defining a branch cut between $k_1$ and
$k_2$ for the square root term in appearing in
Equation~\ref{eq:dispersion_relation_combined}.

%------------------------------

\section{Non-linear dynamics with $D=0$}

Equation~\ref{eq:fixed_point_formulation} defines $\theta$ as a fixed point
\begin{equation}
\theta = G\left[\theta\right] = \frac{1}{ab}  \left(\frac{\partial}{\partial t} + \frac{\partial}{\partial x} +
  b\right)\left( \frac{\partial\theta}{\partial t} - f(\theta) +
  a\theta\right)
  \label{eq:fixed_point_functional}
\end{equation}
of the iteration
\begin{equation}
  \theta_{k+1} = G\left[\theta_{k}\right]
  \label{eq:fixed_point_iteration}
\end{equation}
This iteration will converge if $G$ is a contraction mapping on an appropriate space.

\begin{equation}
\begin{split}
  ab\left(\frac{G\left[\theta_1\right]+G\left[\theta_2\right]}{2}\right) & = 
  \left(\frac{\partial}{\partial t} + \frac{\partial}{\partial x} +  b\right)
  \left(\frac{\partial}{\partial t} - a\right) \left(\frac{\theta_1+\theta_2}{2}\right) - \\
  & \quad \left(\frac{\partial}{\partial t} + \frac{\partial}{\partial x} +  b\right)\left[\frac{f(\theta_1)+f(\theta_2)}{2}\right] 
\end{split}
\end{equation}
Now,
\begin{equation}
\frac{f(\theta_1)+f(\theta_2)}{2} \le f\left(\frac{\theta_1+\theta_2}{2}\right)
\end{equation}
%------------------------------

\bibliographystyle{unsrt}
\bibliography{biotic_transport}

\end{document}
